%/*
% * SPDX-FileCopyrightText: 2021 Stefan Begerad <stefan@begerad.de>
% *
% * SPDX-License-Identifier: GPL-3.0-or-later
% */
%include preamble
%/*
% * SPDX-FileCopyrightText: 2021 Stefan Begerad <stefan@begerad.de>
% *
% * SPDX-License-Identifier: GPL-3.0-or-later
% */
%define outer theme (color and background)
\usetheme{Warsaw}
%\usetheme{Boadilla}
%\usetheme{Copenhagen}
%\usetheme{Goettingen}
%\usetheme{Szeged}

%define color of theme
%\usecolortheme{beaver}
\usecolortheme{seahorse}

%add the following package to use internal links
\usepackage{hyperref}

%add the following package to use tables
\usepackage{colortbl}

%add the following package to use code listings
\usepackage{listings}

\title[Dede]{Designated Driver (Dede)}
\subtitle{Echtzeit-Karte: offen, unabhaengig, universell}
\author{Stefan Begerad}


%define the document
\begin{document}

%define title page
\begin{frame}
  \titlepage
\end{frame}

\begin{frame}{Synonyme}
  \begin{itemize}
  \item<1-> Open Source
  \item<4-> Open Source Software (OSS)
  \item<4-> Free Libre Open Source Software (FLOSS)
  \item<4-> Free Open Source Software (FOSS)
  \item<2-> Free Software
  \item<3-> Libre Software
  \item<5-> \textcolor{blue}{Freie Software}
  \end{itemize}
\end{frame}

\begin{frame}{Lizenzen}
\begin{itemize}
\item SPDX: https://spdx.org/licenses/
\item GNU:
\begin{itemize}
\item https://www.gnu.org/licenses/license-list.html
\item https://www.gnu.org/licenses/license-recommendations.html
\end{itemize}
\item Chooselicense: https://choosealicense.com/
\item Joinup: https://joinup.ec.europa.eu/collection/eupl/solution/joinup-licensing-assistant
\item OSI: https://opensource.org/licenses/category
\end{itemize}
\end{frame}

\begin{frame}{Vier Freiheiten}
\begin{itemize}
\item \textcolor{blue}{Redefreiheit versus Freibier}\pause\\
\item Freie Software bezieht sich auf \textcolor{green}{Freiheit(bspw. Redefreiheit)} nicht auf \textcolor{red}{Preis(bspw. Freibier)}.\pause\\
\item Freie Software garantiert vier Freiheiten. Ist eine abwesend, handelt es sich um proprietäre Software.
\begin{itemize}
\item Verwenden: Freie Software darf für jeden Zweck und frei von Beschränkungen benutzt werden
\item Verstehen: Freie-Software-Code darf beliebig studiert werden
\item Verbreiten: Freie Software darf praktisch kostenfrei kopiert und weitergeben werden
\item Verändern: Freie Software darf beliebig modifiziert werden
\end{itemize}
\end{itemize}
\end{frame}

\end{document}
