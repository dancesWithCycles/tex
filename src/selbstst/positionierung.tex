%/*
% * SPDX-FileCopyrightText: 2021 Stefan Begerad <stefan@begerad.de>
% *
% * SPDX-License-Identifier: GPL-3.0-or-later
% */

\begin{frame}{Positionierung im Wettbewerb}
  Voraussetzung fuer Positionierung:
  \begin{itemize}
  \item Gute Kenntnis vom Wettbewerb!
  \end{itemize}
  Allgemeine Fragestellung:
  \begin{itemize}
    \item Was macht mich im Hinblick auf meine Zielgruppe besonders, wenn ich mich mit dem Wettbewerb vergleiche?
  \end{itemize}
  
\end{frame}

\begin{frame}{Positionierung im Wettbewerb}
  \begin{itemize}
  \item Wo auf dem Markt fuer meine Dienstleistung soll sich mein Produkt einordnen?
  \item Was ist die direkte Konkurrenz?
  \item Falls es keine direkte Konkurrenz gibt: Was ist die Vergleichsgruppe?
  \item Was und auf welche Weise wird angeboten?
  \item Wie unterscheide ich mich in Dienstleistung und Preis von der Konkurrenz?
  \item Sehe ich Wettbewerb regional oder ueberregional(hoehere Spezialisierung)?
  \end{itemize}
\end{frame}

\begin{frame}{Positionierung im Wettbewerb}
  Ergebnis:
  \begin{itemize}
  \item Die Wettbewerbsanalyse vergleicht tabellarisch die Merkmale
    \begin{itemize}
    \item Wettbewerber, USP, Erfahrung, Kontakte, Honorar, Mein Nachteil und Mein Vorteil.
    \end{itemize}
  \item Weitere Positionierungsmerkmale:
    \begin{itemize}
    \item Alter, Anbegot, Ausbildung, Beziehung, Eigene Story, Erfahrung, Honorar, Innovation, Kontakte, Kooperation, Methode, Persoenlichkeit, Service, Spezialthema, Titel, Qualitaetsnachweise, Auszeichnungen, Zertifikate, Zielgruppensprache
    \end{itemize}
  \end{itemize}
\end{frame}
